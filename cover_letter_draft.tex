% \documentclass[a4paper,10pt]{letter}
\documentclass[11pt, addrfooterall ]{if_letter_2013}

% \usepackage[pdftex]{color,graphicx,hyperref}
% \usepackage{fullpage}
% \usepackage{setspace}
% \usepackage[utf8]{inputenc}
% 
% \onehalfspacing
% \setlength{\parskip}{0.3cm plus4mm minus3mm}
% 
% \newcommand{\todo}[1]{{\color{red} \texttt{<{#1}>}}}
% \newcommand{\cpnote}[1]{{\color{blue} {{#1}}}}
% \newcommand{\email}[1]{\href{mailto:{#1}}{{#1}}}

\begin{document}

\begin{letter}{
% \aquieninterese
% \apapiitmendoza
% \aconacytGelover
% Dra. Angélica Leonor Gelover Santiago\\
% Directora de Ciencia de Frontera y Secretaria Técnica del Programa Presupuestario F003
PLOS ONE\\
To whom it may concern
% \asecretariomostalac
% Dra. Maria del Carmen de la Peza Casares\\
% Directora Adjunta de Desarrollo Científico y Secretaria Técnica del FONCICYT\\
% CONACYT
}
\opening{Dear Editor,}

Please find attached the manuscript ``{\it Statistical analysis of word flow
among five Indo-European languages}'', which we submit for consideration in
PLOS ONE as a Research Article. 

This paper studies the flow of words across five different languages (English,
French, German, Italian, and Spanish) using Google n-gram data set and a simple
systematic algorithm that allows to consider large amounts of data.  This allow
us to quantify the influence of one language over others, considering both the
number of words that have migrated and also its importance, as quantified by is
usage frequency, in the receiving language.  Interestingly e find that most
migrant words can be aggregated in semantic fields and associated to historic
events.
Previous studies of language dynamics have been numerous (we have in fact 
contributed in PLOS ONE (2015) with one such studies), specially with the
aforementioned data provided by Google. Other studies have also studied 
word flow (via {\it loan words}) but, to our knowledge, have not used such a
large data set and thus have not offered the statistical analysis that
we here provide. 


Since the primary tool set for this analysis derives from the interdisciplinary
study of complex systems, as potential editors, you might consider:
\begin{itemize}
\item César A Hidalgo
\item Roberta Sinatra
\end{itemize}
We are fine with any reviewer that the editor considers appropriate.
{\bf Dicen que ``
List any opposed reviewers'' por eso pongo esa frase rara. }


Finally, we would like to add that one of the authors (J. Flores) past away about a year
ago, for which an explicit approval for the final version of the manuscript is
not available. However, due to his important contributions we still think that
he must be listed as an author. 

Thank you very much for your time and consideration.



% 
% Carlos Pineda\\
% Robert Gray Dodge Professor of Network Science\\
% Director, CCNR--NEU\\
% Email: a.barabasi@northeastern.edu Website: www.BarabasiLab.com\\

\closing{With kind regards,}

\end{letter}
\end{document}

%paragraph 1: intro (~5 lines)
%one-line description of the paper
%to do: potential recommendation by someone
%\todo{... recommended directing the submission to you.}
%So what: This allows us to model, predict, and control ranking lists more accurately than before.
%paragraph 2: background and challenge to tackle (~7 lines)
%\cpnote{Siento que acá le quita un poco al paper, pues solo serían intenresantes los rankings que implican acceso a recursos. O quizá se pueda ampliar el concepto de recursos para cubrir hienas (las hembras en este caso son un recurso), o simplemente el presitigio.}
%key finding 1 (~7 lines)
%\cpnote{Iteramos la idea. Croe que está mal expuesta, pues hayevidencia de un comportamiento universal, mediado por un unico parametro.}
%In high-flux rankings only the top of the list is stable, while in low-flux rankings the top and bottom are equally stable.
%\cpnote{Se me hace marginal no digno de la carta de presentación}
%\cpnote{En aras de la precision, creo que faltaria mencionar replacement, pero tambipen se me hace que el mecanismo son los vuelos de levy que causan el drift} Gerardo: Por el momento lo dejo asi
%last paragraph: position of work in literature and referees
%paragraph 3: again, description of paper in 2 lines
%key finding 2 (~7 lines)
%Our findings have implications for the theoretical foundations of evolutionary
%biology and evolutionary computer science. In open systems in contact with
%their environment, there is evolutionary advantage in preserving essential
%elements (to maintain {\it robustness}), while allowing for fast exchange of
%less crucial components (conferring {\it adaptivity}). Such a balance has so
%far only been replicated by fine-tuning critical parameters in dynamics with
%homogeneous timescales. We find, instead, that many systems in nature and
%society have varying timescales of rank dynamics: ``slower'' elements provide
%robustness, while ``faster'' elements give adaptivity. Our ability to capture
%this behavior with basic mechanisms of rank flux, irrespective of domain,
%suggests that complex systems may need much less to be evolvable than
%previously thought.
%\todo{referee list, when finalized:}
%\todo{More potential referees:}
%\begin{itemize}
%\item Petter Dodds, U Vermont \todo{ranks in complex systems, rank flux in language: Pechenick et al, J Comput Sci 2017}
%\item Mason Porter
%\item Mark Newman
%\item Alex Arenas
%\item Matjaz Perc
%\item JFF Mendes
%\item Adilson Motter
%\end{itemize}
%\todo{referee list proposal, unranked}
%\begin{itemize}
%\item Michael Batty, UC London. \todo{rank dynamics of city size in history via rank clocks: Batty, Nature 2006.}
%\item Eduardo Altmann, U Sidney \todo{frequent words change less than less frequent words: Gerlach et al, PRX, 2016.}
%\item Renaud Lambiotte, U Oxford \todo{rank-based model for human mobility: Noulas et al, PLoS ONE, 2012.}
%\item Serguei Saavedra, MIT \todo{nodes contribute heterogeneosuly to robustness in networks: Saavedra et al, Nature 2011.}
%\item Christopher Danforth, U Vermont \todo{ranks in complex systems, rank flux in language: Pechenick et al, J Comput Sci 2017.}
%\item Santo Fortunato, U Indiana \todo{rankings of scientists: Radicchi et al, PRE 2009.}
%\item Sune Lehmann, TU Denmark \todo{mobility, ranks of most visited locations: Alessandretti et al, Nat Hum Beh 2018.}
%\item Guido Caldarelli
%\end{itemize}
