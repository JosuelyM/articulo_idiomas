\documentclass{article}
\usepackage{xcolor}

\begin{document}
	
Reviewer 1: \textbf{The manuscript “Statistical analysis of word flow among five
Indo-European languages” presents an approach to describe how words have flowed
from one language to another over the course of at least a century. They use
data from Google Books N-gram to describe these transitions. It is an
interesting topic but I feel that there are some severe limitations in calling
this a statistical analysis and the conclusions that come from it.}

We thank the reviewer for his comments. We have changed the wording so as not to call this 
a ``statistical analysis'' and have provided all necessary details and data to support all
conclusions. We now answer all points raised by the referee. 

\begin{enumerate}

\item % HECHO No le gusta que digamos statistics {{{
\textbf{The abstract does not adequately describe what the reader
will find in the paper. The title and abstract both make it sound like there
will be statistical methods and results to discuss, yet there is not a true use
of statistical analyses to be found in this manuscript.}

We have changed the title of the paper from 
``Statistical analysis of word flow among five Indo-European languages'' to 
``Analysis of word flow among five Indo-European languages'' and modified the abstract avoiding
the use terms related to statistics.
% }}}
% HECHO Cambiar la palabra "model" por "approach" {{{
\item \textbf{On page 2 it is stated that there are two ``models'' used to quantify
the influence of one language on another. However, these are not
statistical models that are being used. They are simply two different
transformations of the data. “Approaches” is a better word than
``model'' although there are other words that would also be suitable,
but model is not a suitable word to describe what is being done.}

Indeed, the word ``approach'' is more accurate than ``model''. We have made 
the appropriate replacements. 
% }}}
% HECHO No quiere una relacion causal entre los hechos historicos y las palabras migrantes {{{
\item \textbf{Later in that same paragraph, after describing the “models” being
used, there is a statement ``...that are responsible for the flow of
words.'' This sounds like a statement of causal inference. However, all
that can really be said is that various things have been identified
that might be related to the flow of words. There is no causal pattern
or analysis within this manuscript.}

We changed ``...that are responsible for the flow of words.'' to 
``that are related to such words.''

% }}}
% HECHO de nuevo statistical por algo diferente {{{
\item \textbf{At the end of the introduction is the following sentence. “Our work
is of a statistical nature, and as such, has its limitations.” I did
not find a statistical analysis in the manuscript so saying that this
manuscript is of a statistical nature is not accurate. There are data
descriptions throughout the manuscript, and some transformations of
the data, but there are no statistical analyses to describe the
patterns or possible relationships.}

We have changed that paragraph to avoid the problematic terms, without changing the main
idea that we wanted to communicate. 
% }}}
% HECHO 5 otro cambio cosmetico {{{
\item \textbf{In the first paragraph of Methodology I would make the following
change. “We removed certain words that did not contributed to the
analysis:” to “We removed the following types of words:”}

Done
% }}}
% HECHO 6 Detalles del data cleaning {{{
\item  \textbf{There is a description of data cleaning and then the following
sentence. ``From this dataset, and after cleaning the data,...'' This
makes it sound like there was additional data cleaning in addition to
just removing certain types of words. This should be more specific in
what is meant by data cleaning.}

We added the details required by the referee.
% }}}

\item At the end of Methodology is a paragraph with some concluding % {{{
remarks on the errors. It would be helpful to know how many errors
like these were found and what proportion of the full dataset
consisted of such errors. Also, I agree that it would be helpful to
have included experts in each language that is being included in this
analysis. \textcolor{red}{codigo}
% }}}

% 8 El segundo idioma mas importante {{{
\item \textbf{mapas de Distribucion de palabras}Below Figure 1 on page 3 is the sentence “Thus, it could be said 
that the second most influential language among the five studies has
been Italian.” Is there a statistical measure or test that can be
pointed towards as validation for this statement? \textcolor{red}{codigo}

El punto 8 ya esta, lo puse como una tabla dentro del archivo del articulo (paginas 5 y 6), ahi puse cuantas palabras migro caa idioma y como se distribuia esa cantidad entre cada reeptor. Ahi lo que falta es como redactar adecuadamente la info de la tabla,

% }}}
\item \textbf{1/2 Ajuste de ley de potencias}The very last sentence of New Words reads “…within statistical % {{{
fluctuations, an asymptotic power law decay with an exponent close to
one.” What is meant by statistical fluctuations? That infers that
there was a statistical model with a measure of variance. Was there a
statistical model fit with these curves to evaluate the fit of a
power-law decay? \textcolor{red}{codigo}

El punto 9 tambien ya esta en la pagina 9 se encuentra una tabla del ajuste de la ley de zipf, los parametros que dan el mejor ajuste asi como un score de la fiabilidad.



% }}}
% 10 {{{
\item  In Accumulated Words is formula 1. The description of the formula 
describes using both frequencies and ranks and it is not clear how
both are involved in the calculation. Also in formula 1, it is not
clear how j and k differ. What does j range to since U(t) is not
indexed by j so there are multiple j values being summed. How does j
relate to k?


% }}}
% 11 {{{
\item \textbf{Histogramas de distribucion de retornos }In the English section under Accumulated Words it states that “the 
use of English in French and Spanish has increased steadily in the
last century whereas in Italian, it has maintained a constant level.”
This is an observational description of trends seen in the data and
there is not a statistical model and analysis to evaluate if trends
are constant or increasing at some rate. It would be nice to include
those trend rates. The same is true for the French, German, Italian,
and Spanish sections. \textcolor{red}{codigo}


Para el punto 11 calcule unos histogramas (estos estan como imagenes en la
carpeta datos\_art/correcciones) donde se ve el ratio al que crecen las graficas
de la figura3. Mi intencion con los histogramas era ver si el ratio era
constante o si ese se cargaba mas a los positivos( idiciaria que el idioma A
siempre crece en el B) o hacia los negativos (A decrece en B). De aqui ya no le
redacte más, de los datos que generan los histogramas se me paso ya que me los
habias pedidido como txt. Estos te lo mando el fin de semana y te los agrego al
git.

% }}}
% 12 Under rank diversity... {{{
\item Under Rank Diversity, why not just use the counts rather than this
Rank Diversity measure. It seems that the figures shown of Rank
Diversity could also be shown using the actual counts. Since the
actual counts are simpler to understand, then it would be easier to
describe the results. If the Rank Diversity does add a detail and
interpretation that is not feasible by using the counts, then this
justification should be included.


el punto 12 de la diversidad de rango, esta medida la sacamos ya que queriamos ver que si se mantenia el comportamiento de diversidad para un grupo mas pequeño, ya que en los articulos pasados tambiens e calculó. Tambien Carlos G sugirio que la hicieramos ya que en lo que hizo sergio y en el que hizo Jose antonio de deportes y juegos tambien se habia calculado, y era como algo "natural" volverla a calcular para un grupo de dtaos diferentes. En esta parte no se como  argumentarle eso de darle conyinuidad a trabajos previos.
% }}}
% {{{ 13
\item  In Figure 5 and equation 2, the year is changed to log base 10. 
Please explain the rationale for this and make it clear to the reader
that this is being done and why it is being done.


El 10 y 13 es un problema con la redaccion de la formula. Aqui si ya no creo poder moverle.

% }}}
% {{{ 14 
\item  In Figure 5 and equation 2, were any other functions considered 
besides the cumulative Gaussian function? There are other sigmoid
shapes that could be used so why was this one chosen? Also, what is
meant by mu and sigma were obtained with a linear regression. This is
a nonlinear function so it would be nonlinear regression to estimate
those parameters and were they estimated using nonlinear least squares
or maximum likelihood or something else? Then for the results of this,
why can that conclusion of migrant accumulated words in the middle and
high ranks are the ones that tend to change their position. How
exactly was this determined because it is not obvious from the
equation or from the figure. \textcolor{red}{codigo}

El punto 14 va relacionado con lo anterior, ya que como habian calculado funciones cumulativas gaussianas en los trabajos previos, entocnes era normal buscar la misma distribucion.


% }}}
% 15 {{{
\item  Although Google Books N-gram data are available for download, the
specific cleaned data used in this analysis should be made available. \textcolor{red}{anexo de palabras funcionales eliminadas??}


El 15 es el codigo que genera todos los reultados. Yo ya no encontre el codigo, lo que tengo son los datos que generan las graficas y las listas de palabras ordenadas por decada. 

% }}}
\end{enumerate}

\end{document}
